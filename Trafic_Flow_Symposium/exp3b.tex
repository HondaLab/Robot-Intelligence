直線と円形両方あり,一回じゃなくて,ぐるぐる回って複数の対面走行できるように,楕円コースで実験する.コースの中(青い部分)にロボットをランダムに配置して速度0からほぼ同時にスタート,約8分間実験する.ロボットたちがtof距離センサーでコースの壁までの距離を測って,非線形感覚運動写像モデルにより走行する.半分のロボットが右回り($b_{\rm L}>b_{\rm R}$),残りのロボットが左回り($b_{\rm L}<b_{\rm R}$).コースの青い部分の中央に(黒い線)でコースの長さを測る,コースの長さ($L$)は7.32$m$,今回ロボットの台数($N$)は8台.コースの長さを固定し(線密度($\rho$)を固定する(式5)),両側の壁を広まったり,縮まったりして,コースの幅($w$)が$43cm$,$49.5cm$,$56cm$,$62.5cm$,$69cm$を変化し,実験する.

\begin{equation}
\rho = \frac{N}{L}
\end{equation}
\begin{equation}
Q = \left| \frac{n}{wT_{\rm sd}} \right|
\end{equation}

\begin{figure}[h]
    \begin{minipage}{0.48\linewidth}
        \centering
        \includegraphics[width=0.9\linewidth]{course3.jpg}
        \caption{コース}
    \end{minipage}
    \begin{minipage}{0.48\linewidth}
        \centering
        \includegraphics[width=1.0\linewidth]{Oval_h2.eps}
        \caption{コースレイアウト}
    \end{minipage}
\end{figure}



図5の赤い線をベースとして,ロボットが左から右へ線を通過したら流量+1,右から左へ線を通過したら流量-1.図6の横軸が時間(0から8分間),縦軸がロボットの位置角度$\theta$(図5),ロボットが赤い線から反時計回りで黒い線まで,$\theta$が0から$\pi$に変わる.赤い線から時計回りで黒い線まで,$\theta$が0から$-\pi$に変わる.横軸の時間に従って,紫色の線が縦軸の0を交差すると,計測線(赤い線)を通過したと認めて,台数($n$)を計測する.$T_{\rm sd}$はone direction flow 状態になる時間,$w$がコースの幅(単位:$m$),$Q$が流量,式6で流量を計算する.

