実世界で,蜂,アリなどの昆虫が簡単な行動メカニズムによって,複雑な群れ行為ができる.
また,大きな交差点などにおいて,人間は密度が高くても,会話なしで,
ぶつからないようにスムーズに対面歩行ができる.

池田ら\cite{ikeda16}は,非常に密度の高い自己駆動粒子の対面流において異方性を考慮
することによってレーン形成が生成することを見出した.

本論文では,我々は原生生物レベルの反応行動のための知能を持つ走行ロボットを開発した.
コース幅が限られたコースにおける,その走行ロボットの対面走行を実験的に観察する.

今回,最大8台使われるロボットを時計回りと反時計回りの2つグループを分けて
楕円コースでの対面走行を実験を行った.
コース幅を変化させ,ロボットの振舞いを観察した.
ロボットが,1方向走行流になるまでの時間と流量などロボットの基本的な走行情報を測定した.
