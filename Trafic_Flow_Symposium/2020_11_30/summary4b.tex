本研究で用いたロボットは最適速度ロボット\cite{yamada19}のような,
他のロボットを追いかける機能はついてない.
それにもかかわらず,
単純な障害物回避アルゴリズムによって,
最終的に1方向走行流の状態になる傾向があるとわかった.

対面走行流から1方向走行流への転移が起こるまでの時間を観測した.
コース幅が十分に大きくなると,その転移時間は一定となる傾向を見出した.

また,コース幅が十分に大きくなると流量も一定になる傾向があることを観測した.

実世界のアリ,蜂などの昆虫の匂いで作られたコースのような空間の中,
単純な障害物を避ける行為で自動的にレーン形成したことと類似の現象が観測されたと考えられる.

今後の実験で,ロボットの線密度を増やすことによって,より実世界の人間などの対面流行動と類似の
行動が観測されると予測される.

