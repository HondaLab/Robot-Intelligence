本研究では,昆虫や人間の対面走行行為にどんな知能を持っているかを解明するため,ラズパイを基づいてtof距離センサー3つ搭載され,ハイパボリックタンジェント(tanh)関数でセンサーからもらった距離デーダーをロボットの左右モーターの出力に写像して,障害物を避ける走行ロボット開発した.そのロボットを使って,楕円コースで異なる台数のロボットの対面走行実験をして,ロボットの時速,流量とone direction flowになるまでの時間を測定した.結果として,時間とともに,どんな初期配置でも,結局,全てのロボットが同じ向きで走り(one direction flow)になる傾向があると観測した.one direction flow になる時間が幅の拡大に従って,減少していく.流量が山登りみたいに増加して幅が49.5$cm$から減少していくとわかった.
