実世界で,蜂,アリなどの昆虫が簡単な振舞いや匂いで複雑な群れ行為ができる.
大きな交差点で,人の密度が高いでも,皆は会話なくて,ぶつからないようにスムーズに対面走行ができる.
その中に一体どんな知能が持っている,どのぐらいの知能が必要だと知りたいので,
我々は原生生物レベルの感覚と運動直接関連する感覚運動写像[1]での障害物避ける振舞い,
反応行動レベルの知能を持つ走行ロボットの対面走行を始めにその知能を解明するである.
本稿では,tanh関数を使い非線形感覚運動写像をモデルとし,
ラズパイを基づく障害物を避ける走行ロボットを開発した.
その走行ロボット(今回,最大8台使われる)を右回りと左回り(変曲点係数bで制御する)の2つグループを分けて楕円コースでの対面走行を実験して,
変曲点の係数(b)と初期配置を変化させ,ロボットの振舞いを調査して,
ロボットたちの時速,同じ流れになるまでの時間と流量などロボットの基本的な走行情報を測定し,one direction flow 状態を観測した.
