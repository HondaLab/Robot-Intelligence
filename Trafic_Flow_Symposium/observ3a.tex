\begin{enumerate}
\item $b_{\rm L}>b_{\rm R}$:ロボットが右曲がり易い($b_{\rm L}$の方が大きいので,同じ距離$x$で$b_{\rm L}$に対応するtanh関数の変曲点が横軸の正方向に多く移動して,$x-b_{\rm L}$が小さくなる,方程式3により,右のモーターが左のモーターより先に速度を減少するので,ロボットが右曲がりやすいである)
\item $b_{\rm L}<b_{\rm R}$:ロボットが左曲がり易い
\item 初期配置:ロボットの位置はランダムで,グループ2のロボットが左回り,グループ2のロボットが右回り.
%\item 密度:ロボット台数/コース面積
\item $T_{\rm sd}$:全てのロボットのスタートから全てのロボットが一方向走行するまでの所要時間
%\item 渋滞時間割合:$T_{\rm sd}$に対して,渋滞時間の割合
%\item 同じ流れ方向:同じ流れになる時間後の
%          一方向走行の方向
\end{enumerate}


ロボットが一つずつ5周回って,時速を計算する($v=\frac{5*L}{t}$,$v$:時速,$L$:コースの長さ,$t$:5周回る時間).時速の平均値$\bar v$:787.52$m/h$;時速の分散($s^2$):2640.3;標準偏差($s$):51.384

