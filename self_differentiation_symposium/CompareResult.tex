流量を比較するため,ランダムと渋滞,2種類の初期配置でニューラルネットワークより,感覚運動写像より約5回実験した.
図\ref{random_result}と図\ref{crowd_result}は2種類の初期配置での実験回数($x$軸)と流量($y$軸)の関係図です.

点線がニューラルネットワークより走行の結果,実線が$b=270,270$の感覚運動写像より走行の結果,一点鎖線が$b=360,260$の感覚運動写像より走行の結果です.
ニューラルネットワークの流量が常に感覚運動写像の流量より高い,変動も平穏だと見られる.

図\ref{compare_result}はランダムの初期配置(実線)と渋滞の初期配置(一点鎖線)で約5回の実験の流量平均値グラフです.
$x$軸はアルゴリズム種類,$y$軸は平均流量.この図からニューラルネットワークの平均流量も感覚運動写像より高くて,異なる初期配置に対して平均流量があんまり変わらない.
$b=270,270$の感覚運動写像の平均流量が初期配置によって差が観察されて,それは実験の回数が不足かなと考えて,今後,実験の回数を増やす必要があると思う.


\vspace{-1mm}
\begin{figure}[!ht]
    \centering
    \includegraphics[width=0.9\linewidth]{result_diagrim_rand.eps}
    \caption{ランダムの初期配置で実験回数により流量の変化図}
    \label{random_result}
\end{figure}


\vspace{-1mm}
\begin{figure}[!ht]
    \centering
    \includegraphics[width=0.9\linewidth]{result_diagrim_crow.eps}
    \caption{渋滞の初期配置で実験回数により流量の変化図}
    \label{crowd_result}
\end{figure}


\vspace{-1mm}
\begin{figure}[!ht]
    \centering
    \includegraphics[width=0.9\linewidth]{mean_Q.eps}
    \caption{異なるアルゴリズムの平均流量の比較}
    \label{compare_result}
\end{figure}
