1次元画像データ認識ニューラルネットワークより8台ロボットが縞模様のコースで対面走行ができると確認した.
感覚運動写像と比べて,方向転換教えなくても,対面走行を維持する能力が高い.

本文説明した対面走行維持できるニューラルネットワークの学習結果は,著者が何十回練習して,経験を踏まえ,ラジコンする時,遠くから曲がる,近づいて曲がる,真正面にロボットや壁ある時後退することをわざわざ人間の意識持ってロボットに教えて収集した教師データの学習結果です,研究室の他の人がラジコンして収集した教師データも学習して,自律走行の質もそれぞれだ.それに,コースの壁の代わりに,別のもの(ダンボール,雑誌,サンダルなど)を配置して,収集した教師データの学習結果では自律走行ができないことが観察された,その原因として,画像を列ずつ足し算してもらった1次元画像データが障害物と通路の特徴を失って,区別できなくなると考える.

今後の展望として,色んな教師データの違いと特徴をを分析して,どんな教師データがあれば対面走行維持できるを解明すると教師データの質を評価方法を開発する必要がある.
また,画像エントロピーで環境の複雑度を評価して1次元画像データの限界を解明すると考える.

