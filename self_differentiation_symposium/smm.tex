感覚運動写像とは,センサー値を変数とする関数によってモーターの出力を決定することであり,
その瞬間のセンサー値だけを使う,最も単純な反応行動のための知能の一つである\cite{asada}.
本研究では,非線形感覚運動写像モデル(式(\ref{eq:mR})と式(\ref{eq:mL}))が使われている.

3つの距離データを相乗平均する得られた$x_{\rm L}$と$x_{\rm R}$を式(\ref{eq:mR})と
式(\ref{eq:mL})に代入して,ロボットの右モーターの出力($m_{\rm R}$)と左モーターの
出力($m_{\rm L}$)を計算する.
$b$は$\tanh$曲線の変曲点であり,実験ではロボット曲がるの反応距離或いは障害物をぶつからない安全距離である.
今回の実験のパラメーターは
$\alpha=35\%$とする.
すなわちロボットは最高速度の70\%の速度で走行する.ニューラルネットワーク教師データ収集ラジコンする時の最高速度と同じです.
$\beta_1=0.004$,
$\beta_2=10$,
$c=0$とする.
詳しい内容は参考文献\cite{li}を参考してください.

\begin{eqnarray}
\begin{aligned}
  m_{\rm R} = &\alpha \tanh(\beta_1(x_{\rm L} - b_{\rm L})) + \\
        &\alpha \tanh(\beta_2(x_{\rm L} - b_{\rm L})) + c
 \label{eq:mR}
\end{aligned}
\end{eqnarray}

\begin{eqnarray}
\begin{aligned}
  m_{\rm L} = &\alpha \tanh(\beta_1(x_{\rm R} - b_{\rm R})) + \\
        &\alpha \tanh(\beta_2(x_{\rm R} - b_{\rm R})) + c
 \label{eq:mL}
\end{aligned}
\end{eqnarray}
